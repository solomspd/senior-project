\documentclass{article}

\usepackage{float}
\usepackage{dirtytalk}
\usepackage{graphicx}
\usepackage{biblatex}

\addbibresource{bibliography.bib}

\author{}

\title{Proposal}

\begin{document}

\maketitle


\section{Introduction}
Although reverse engineering a binary program has a rather mature ecosystem of tools such as Ghidra and Binary Ninja, they all still involve a large amount
of manual time and effort to bring back some sense of structure and human readability to the assembly code. This is mainly due to their dependence on pattern matching and other rule-based approaches which introduces many limitations to traditional decompilers including poor scalability and development.
Since the purpose of decompilation is crucial to multiple cybersecurity domains such as malware analysis and vulnerability discovery, the more this process is enhanced, the more we can offer faster results to analysts and reverse engineers.

\section{Problem Description}
\subsection{Summary}
One of the core conceptual problems with decompilation is that although you can deterministically compile code into assembly, the reverse is untrue; there are countless forms that code can take that produces the same Assembly.
To this end the deterministic nature of rule based decompilers fails to pick which of the countless routes to take.
Machine learning offers a heuristical approach to settling these uncertainties, allowing us to get as close as possible to how a human have written it.
Furthermore, x86 architecture has been given plenty of the focus in recent papers so this leads our attention to seeing what would happen if we use these techniques on other architectures, namely android and its Java virtual machine.
Moreover, besides the typical compilation problems that are present in any decompiler(e.g. syntactic distortion, semantic incorrectness, and major difficulty with code readability), android apps decompilers show a considerable failure rate in recovering programs as well as a major bias towards a specific goal/usage of the decompiler.
In addition, android decompilers perform differently across different applications and the output of the decompiled code is heavily dependent on the compiler used to compile the original source code.
We believe that the more expressive nature of java byte code has the potential to open up for us better prediction along with easier extrapolation of underlying structure due to java’s more strict object oriented paradigm.
Also this is to provide tools to combat the current rise in security concerns on the mobile platform.

\subsection{Proposed Solution}
The goal of this project is to propose a new platform for credible Android apps ML-based decompilation to raise the accuracy of code recovery.
Our proposed project adopts and improves on major successful models that dramatically enhanced the performance of desktop apps decompilation.
Hence, beyond simply migrating well established solutions from \verb|C| binaries to bytecode, we also propose a 3 phase solution to addressing the decompilation problem in Android applications, each phase leveraging a different approach to extracting information from the cryptic binary.

\section{Literature Review}
Machine learning based decompilation has been explored in a number of research projects before.
Most of the previous work that addressed this problem focused solely on recovering high level \verb|C| based on \verb|x86| architecture.
Thus, besides reviewing the literature dedicated to ML-based system and techniques for decompiling \verb|C| binaries, we also present a complete review for all the popular platforms and applications used for decompiling Android applications. \\\\


\subsection{Binary enrichment}

\subsubsection{Debin}

Debin, a prediction system, aims to deal with the stripped binary that contains low-level information, which are often unknown due to optimization purposes, or even worse to hide malicious and vulnurable code.
To recover these stripped information such as variables and types, Debin tries to automate this process with machine learning techniques instead of relying on rule based techniques. In other words, Debin was trained on thousands of non-stripped binaries and then used to predict properties of meaningful elements in unseen stripped binaries.
Debin's support for binary files extends to three achitectures: \verb|x86|, \verb|x64|, and \verb|ARM| with high levels of precision.
Debin mainly uses two probabilistic methods to implement the recover the variables, which is the Extremely randomized Tree, as well as linear graphical model for the prediction on the extracted program's elements such as their names and types.

Basically, Debin follows five main steps to produce the improved and enhanced binary file. Firstly, the stripped binary file is taken as input and then lifted from assembly code into an intermediate Binary Analysis Platform (BAP-IR). The reason for this transition is have a high level form of the sementacis available, generalize the
syntax among the three achritectures Debin targets, and a better understanding of the logic instructions. BAB-IR preserves the raw instructions semantics, showes explicitly the operations on machine states, and also recognizes the function boundaries via its ByteWeight component to obtain the code elements.
The next step would be to extract two type of data from the intermediate, which are the known and unknown elements. The unkown elements are the ones that have been lost in the compilation process due to optimization's stripping. In contrast, the known elements are the already present properties in the binary
code and no need to infering any its information.Examples of known elments would be Dynamically linked library (DDL) functions, flag, instruction, unary operator, constant and location nodes. Also, temporarily allocated registers and memory offsets are treated as known nodes and do not need any name or type prediction.

After acquiring the known and unknown elements, a dependacy graph is built and formed with nodes and their relationship with each other as edges.
Moving on to the next crucial step, Debin starts to infer the unknown elements through the pobablistic method, Maximum a Posterior (MAP).The relationship between elements is represented in the form (a,b,rel). These three words represents node \verb|A| that is connected to node \verb|B| with a relationship between them rel. Those relationships can vary between functions, variables, types, and factor relationships.
Finally, after the predictions are applied on the unkown elements, an updated binary file is released as output and ready with its improved debug information.

Debin dataset consisted of \verb|9000| non-stripped binary excutables that were originally written in pure C language, \verb|3000| for each of their targeted architectures. They gave 2700 of the binary files as training data and left the remaining 300 binary files for testing and prediction purposes.
Also, Debin uses a binary classifier to better evaluate the updated binary file's accuracy based on the following formula, Accuracy = ${\frac{|TP|+|TN|}{|P|+|N|}}$. After going through all 9000 binary files and evaluating them, variable recovery reached 90.6\%, type recovery reached 73.8\%, name prediction accuracy up to 63.2\%, and structured prediction up to 63\% precision.

Debin's major key limitations exist in predicting the contents of struct and union types. Also, the trained-based output of different compilers works well in predicting symbols omitted in compilation, yet it does not perform well with the initial use of obfusication or human-based written assembly.


\subsubsection{Probabilistic Disassembly}
Analyzing and reversing software have many applications; however, it is a very challenging process because the source code usually is not present. The first problem is how to disassemble the software accurately. This process is highly complex because of the complexity because of the diversity in the compilation and optimization techniques. The paper discusses two popular disassemblers: linear sweep disassemblers and traversal disassemblers. linear sweep follows the byte order, whereas traversal disassembling follows the control-flow of function calls and jumps. The problem with linear sweep is that it introduces many false positives and even false negatives, especially when data and code interleaves. On the other hand, traversal disassemblers suffer indirect control flow like in switch-case statement. \\

\noindent The paper also discusses \textbf{Superset Disassembly}. It is the state-of-the-art technique in rewriting/instrumentation. The idea behind this approach is to consider every starts an instruction, called superset instruction. Consecutive superset instructions can share common bytes. Superset disassemblers have no false negative but must have a bloated code body because of the large number of superset instructions that are false positives.\\

\noindent The paper then discusses the approach they proposed about probabilistic disassembling. This approach inherits the strengths of superset disassembling that it produces no false negatives. This is because true positives exhibit a lot of hints indicating that they are true instructions. However, hints are not certain; thus, false positive instructions as well have a low chance of exhibiting the same features. \\

\noindent In x86, part of a valid instruction may be another valid instruction, and also two valid instruction could have overlapping bodies, and these are called \textbf{occluded instructions}. A problem with occluded instructions is that they can be cascaded. So, if we start from the wrong place, many following instruction are occluded. However, this cascading is highly unlikely. The good point that if one of the sequences is the true positive, then the occluded sequences quickly converge with the true positive. The authors also noticed that the suffix of an instruction is likely to be another instruction. This is based on the \textbf{occlusion rule} which states that occluded sequences tend to quickly agree on a common suffix of instructions\\

\noindent The paper then discusses what the authors call \textbf{probabilistic hints.} Simply this is a way of predicting that the analyzed bytes are valid instructions not data. The first is hint is \textbf{control flow convergence.} This is done by analyzing the bytes and finding more than jump to a valid instruction; this usually indicates that those bytes are more likely to be instructions not data. The second hint is the \textbf{control flow crossing}. This happens when there are more than one jump instructions crossing each other. For example, when there are three instructions \emph{inst1, inst2, inst3} and \emph{inst2, inst3} are adjacent to each other, and \emph{inst1} jumps to \emph{inst3} and \emph{inst2} jumps to instruction before \emph{inst1}. Since it is highly unlikely that data bytes can form two control flow instructions with one jumping right after the other, they are most likely to be instructions. The third hint is \textbf{register define-use relation}. A pair of instructions \emph{inst1, inst2} have a register define-use relation when \emph{inst1} defines a values of a register and \emph{inst2} uses it. For example, when a comparison instruction sets a flag and then used by a conditional jump. Those hints indicate that the analyzed bytes are not data, but they don't assure that they are true positives. This is due to the fact that they may be occluded instructions that are part of some ground truth instructions, as they share similar features such as register operands. Fortunately, the occlusion rule tells us that if there is even an occlusion, it will automatically be corrected.

\subsection{Android Decompilers}

\subsubsection{Java Decompiler}
Java IDEs such as IntelliJ and Eclipse include built-in decompilers to help developers analyze the third-party classes for which the source code is not available. Decompilation is the process of transforming bytecode instructions into source code /cite{harrand_java_2020}. An ideal Jdecompiler is one that can transform all inputs into source code and this decompiled code can be both recompiled with a Java compiler, and behaves the same as the original program. However, previous studies that compared Java decompilers found that this ideal Java decompiler does not exist because of the irreversible data loss that happens during compilation.
A decompiler’s capacity to produce faithful retranscription of the original code is evaluated by: Syntactic correctness: when the produced decompiled code is recompilabe with a java compiler without producing any error, Syntactic distortion: the minimum number of atomic edits required to transform the abstract syntax tree (AST) of the original code into the decompiled version, Semantic equivalence to modulo inputs: when the decompiled program passes the set of tests from the original test suite, and Deceptive decompilation: when the decompiler output is syntactically correct but not semantically equivalent to original inputs /cite{harrand_java_2020}.
In this paper, a comprehensive assessment of: syntactic correctness of the decompiled code, semantic equivalence with the original source, and syntactic similarity to the original source, was performed by evaluating eight recent decompilers on 2041 Java classes. The set of decompilers under study were CFR, Dava, Fernflower, JADX, JD-Core, Jode, Krakatau, and Procyon. Each decompiler was developed for different usage, and was meant to achieve different goals. For example, CFR is used for Java 1 to 14 for code compiled with javac, Procyon from Java 5 and beyond and javac, Fernflower is embedded in IntelliJ IDE, Krakatau up to Java 7 and does not support Java 8, JD-Core is the engine of JD-GUI and supports Java 1.1.8 to Java 12.0, JADX targets dex files, and Dava produces decompiled sources in Java and does not decompile bytecode produced by any specific compiler nor from any specific language /cite{harrand_java_2020}.
Each of the eight decompilers was evaluated according to the four characteristics to produce faithful retranscription of the original code that were mentioned above. For syntactic correctness, no single decompiler was able to produce syntactically correct sources for more than 85.7\% of class files in the dataset they used. This implies that decompilation of Java bytecode cannot be blindly applied and requires manual effort. For semantic equivalence and equivalence to sources modulo inputs, five decompilers generate equivalent code for more than 50\% classes /cite{harrand_java_2020}.
Then, they isolated a subset of 157 java classes that no decompiler can handle correctly and merged the results with several incorrect decompiled sources.
After that they merged the results of the incorrect decompilers to produce a version that can be recompiled through a process known as Meta-decompilation. Meta-decompilation is a new approach for decompilation that leverages the natural diversity of existing decompilers by merging the results of different compilers and it is able to provide decompiled sources for classes that no decompiler in isolation can handle. They named their meta-decompiler Arlecchino /cite{harrand_java_2020}.
This paper’s main takeaway is that even the highest ranking decompiler in their study produces correct output for 84\% of classes of the dataset used and 78\% equivalent modulo input. Even their new tool Arlechino can produce semantic equivalence modulo inputs sources for 37.6\% of classes that were not successfully decompiled by the other eight decompilers /cite{harrand_java_2020}.
\subsubsection{Kerberoid: A Practical Android App Decompilation System with Multiple Decompilers}
One of the most notable attempts dedicated towards Android decompilers development is "Kerberoid" []. Kerberoid relies on the meta-decompilation approach to integrate different outputs generated from different decompilers so that it achieves higher accuracy and coverage of the decompiled code. The integration happens through the classification of the decompilers' output and automatic selection of the best partial result from each. Kerberoid does this process on 4 main steps. The first step (Collector) runs the target executable (the input APK file) on multiple decompilers, namely CFR, JD Project, and Jadx and structure these outputs so that they are passed to the next stage (Parser). Different code blocks such as functions, variables, and classes are classified during the second stage of the process (Parser) to be passed to the third stage (Integrator). Overlapping between different code blocks generated from different decompilers is handled during the third phase of the process by comparing the code blocks from different decompilers against each other. Finally, the output from the integrator is passed on to the fourth and final stage of Kerberoid. The final stage (Selector) is based on a machine learning classifier that is trained to select the best code block among multiple potential code blocks produced from multiple decompilers for the same function.\\\\
Using this technique, Kerberoid outperformed existing Andorid apps decompilers that it was assessed against. It managed to recover 75\% of the overall functions as successfully recovered 75\% of the applications in the test set.

\subsubsection{DexFus: An Android Obfuscation Technique Based on Dalvik Bytecode Translation}
This paper describes how obfuscation increases the difficulty of reverse analysis of android applications without changing the semantics of the original code. It also describes how current android obfuscation techniques primarily concentrate on Dalvik byte code obfuscation as the byte code in this case contains much semantic information and obfuscation won’t hinder decompilation that much [].
As mentioned, existing android obfuscation tools focus on Dalvik bytecode obfuscation including encrypting the original DEX file and migrating x86 architecture techniques. The decrypted DEX file will be loaded into memory at runtime and can be obtained via memory dumping attack. Also, the Dalvik bytecode contains too much analytical data which makes it easier to analyze than assembly code [].
This paper then introduces three common android obfuscation systems: Proguard, Android Shelling, Compile-Time Code Virtualization, and Migrate x86. Proguard is integrated into Android Studio, and provides minimal protection against reverse engineering by obfuscating names of classes, fields and methods. Android Shelling works at the DEX obfuscation level. It replaces the origin Dex file with a stub Dex file then encrypts the origin and hides it in the assets directory. Compile-time code virtualization uses automatic tools to transfer code virtualization from DEX level to native level at compile time. The project contains two components, the pre-compilation component for improving the performance, and the compile-time virtualization component that automatically translates Dalvik bytecode to LLVMIR. This translates Dalvik bytecode into VM instructions instead of raw native code and splits the method into small ones, which sacrifices few efficiencies. Finally, Migrate x86 techniques like control flow flattening and instructions substitution to Dalvik byte code. Their work increases the cost of reverse engineering. However, Dalvik bytecode, which can be disassembled back into .smali file, is easier to read than assembly language. Attackers may not spend too much time to understand the obfuscated Dalvik bytecode [].
Then, the authors of the paper introduce their solution which is a proposed system that translates essential methods in the origin Dalvik bytecode to C code and obfuscates it to perform an expressive obfuscation. This tool is called Dexfus. First, dexfus uses Apktool to decompile Dex files in Android applications, then to a C code translator. Then, it hits the JNI call strings and UTF8 strings in the methods and encrypts and replaces these strings with function calls that decrypt strings at execution time. DexFus replaces Hot DVM methods with C methods if it can be translated, to reduce JNI call consumption. It will then insert the code of loading compiled native libraries automatically and use
apktool to repack the modified Dex files and copied native libraries together to an obfuscated APK file. After translating from Dalvik bytecode to C, the code still contains plain-text strings such as JNI methods which are then encrypted by the tool. After encryption, the original plain text string will not appear in the dynamic link library and will be replaced by a function call in the form of GetResource (StringId) [].
Finally, the paper is concluded describing how android applications are facing severe threats like cracking and repackaging, and how Dexfus applies obfuscation on translated C code which provides a higher level of protection than obfuscating the original Dalvik bytecode [].

\subsection{Neural Decompilers}

\subsubsection{Using recurrent neural networks for decompilation}
\noindent One of the very early attempts to generate reasonable results using ML in decompilations systems was introduced by Katz et al. for recovering C source codes from binary snippets using encoder-decoder recurrent neural networks model which acts as a decompiler \cite{katz_using_2018}.
The authors of the paper trained the model to recognize different patterns and properties that exist in human-written source codes so as to generate a similar result when decompiling different binary files.
Their main approach depends on an adaptation of a natural language translation RNN model to translate between different programming language representations using RNN-based encoder-decoder translation system \cite{katz_using_2018}.
In specific, sequence-to-sequence based model is used in \cite{katz_using_2018} to process the input sequence using an encoder RNN, then a hidden state, contained in the model, is used so as to summarize the entire input sequence.
The summarized output of that hidden model is then fed to a decoder RNN which eventually creates the output. \\\\
A major advantage of the proposed platform is its easy extensibility to new programming languages with enough data fed to train the model on that new language.
Authors of \cite{katz_using_2018}, however, focused only on recovering C source codes compiled at optimization level zero which poses a validity threat that their technique might not be usable when decompiling real world projects since most of these projects do not user zero level optimization.
In addition, one of the main limitations of their approach is that it only works for small snippets of binary machine code and fails to recover longer ones making this approach inapplicable to real world code projects due to their lengthy code profiles \cite{katz_using_2018}.
Besides these limitations, the Seq2Seq model used in \cite{katz_using_2018} was found to be not suitable for decompilation problems mainly due to the difference in formats between natural languages and PL which was not accounted for when adapting the model to decompilation problems.\\\\

\subsubsection{Coda: An End-to-End Neural Program Decompiler}
\noindent To address the challenges and limitations highlighted in \cite{fu_neural-based_2019}, in 2019, Fu et al. introduced "Coda" framework to be the first end-to-end neural-based framework for code decompilation \cite{fu_neural-based_2019}. To address the problem of decompilation, the authors divided the decompilation process into two main phases (code sketch generation and iterative error correction) and tackled each phase separately.\\\\
For code sketch generation (first phase of the solution), the authors of \cite{fu_neural-based_2019} use an instruction type-aware encoder and an abstract syntax tree (AST) decoder supported by attention feeding mechanism to convert between the input binary and high-level programming language.  The code sktech generation phase works as follows: different types of instructions including memory, arithmetic, and branch instructions serve as the inputs to the encoder which makes use of e N-ary Tree-LSTM models to encode these different instructions types. A dedicated LSTM model is designated to handle each statement of the input program according to its instruction type. To address the adaptation problems encountered when using natural languages models on programming languages, Coda's abstract syntax tree (AST) is generated according to a tree decoder as it offers multiple advantages and some solutions to the limitations highlighted in \cite{katz_using_2018}. Using a tree decoder and terminal node representation automatically preserves code statement boundary and facilitates sytanx restriction verification. In addition, the connections between the nodes of the tree better highlights the dependency constraints in the program being analyzed. It also makes it easier to mitigate the error propagation problem during the code generation stage. In contrast to \cite{katz_using_2018}, the RNNs model presented in \cite{fu_neural-based_2019} dedicates a separate RNN for each statement type which results in preserving the modular property of the programs (preserving statement boundaries) as well as capturing the control and data dependencies of the program (the connections between the different states of the RNN.  \\\\

The second stage ( Iterative Error Correction) of the proposed method focuses on solving the prediction errors that are unintentionally generated during the code sketch phase. To tackle this problem, authors of \cite{fu_neural-based_2019} develop an iterative an error predictor that works on the output of the autoencoder-decoder of the first stage to increase the accuracy of translation. The output of this predictor indicates the location and error type information in the code sketch \cite{fu_neural-based_2019}. The error correction machine takes this output and relies on confidence scores generated by the EP to prioritize correction strategies that have the potential to increase the accuracy of the generated high-level code \cite{fu_neural-based_2019}.\\

Although the method proposed in \cite{fu_neural-based_2019} outperforms earlier attempts to deal with decompilation as a translation problem, the method also suffers from considerable limitations that affect its performance and applicability to real-life decompialtion problems. These limitations include the decrease in its performance when working on low-level input code buit from complicated ISA such as x86-64 \cite{fu_neural-based_2019} . In addition, Coda framework doesn't provide reliable recovery and does not work properly when complex data structures, control graph, static library are used in the input program \cite{fu_neural-based_2019}. Moreover, the frameowork generates credible recovery results only with short programs with the average code length of 45 and 60 \cite{fu_neural-based_2019}. 

\subsection{Toward Neural Decompilation}

Discovering vulnerabilities and malware analysis begins by comprehending the low-level code comprising the program. Although most Reverse Engineers and Malware Analysts go through this process manually by reverse engineering
the program, it is a slow, time consuming, and tedious task. The problem here lies in the act of manually going through each and every line to try and understand what a program does and how it is done.
Hence, decompilation can greatly improve this manual process by automatically translating the excutalbe binary code to a better visual and readable higher-level code. Not only is Decompilation useful in vulnerability and security analysis, but also in the
easiness of portability to other architectures or operating systems since we are dealing with the source code itself.

Many existing decompilers vary in their decompilation process. For instance, there are those who rely on pattern matching to clarify the relations between the low level and high level structures. However, the failure rate of this method is high when used on sophisticated code and advanced statements
such as the goto. Semantically equivalence may be achieved in comparison to the orginal binary file, yet it's unreadable and non-efficient. Another decompiler method that is worth mentioning is those that are based on neural machine translation (NMT) due to their significant improvement and results in regards to binary to source code translation.
Still these decompilers have their own problems and constraints. For example, a decompiler that used RNN for decompialtion had difficulties relating between the transalted programming languages and our natural language, thus leading to poor results. Hence, the code they generate often is not recompilabe or is not equivalent to the original source code.

Automatic neural deocmopiler is a two phased approach that aims at targeting some of the previously mentioned problems above. The first stage tries to generate a template code snippet that has an equivalent structure including computation to the input file. The second stage would be filling the template code snippet with the programs values to finalize the decompilation process.
Instead of applying the methods of the existing NMT decompilers that work directly on binary files without strong natural langauge knowledge, Automatic nueral decomopiler applies first augmentation with programming-languages knowledge (domain-knowledge). Through domain-knowledge, the NMT translation will by more readable and simpler code in contrast to the original NMT decompilers.
Also, Automatic neural deocmopiler incoperates techniques used in Natural Language Processing (NLP) to better structure the translated programming language.The second phase of the Automatic nueral deocmopiler recieves the template code snippet as input to find the right values for
represent actual code from the template code snippet released from the NMT translation. The second phase verifies that the generated values are correct and relevant, if not then they are replaced using the delexicalization practices learned through NLP. Hence, the process starts from the assembly code, to the NMT mode, to the snippet code result, to the final NLP checking step.

Automatic neural deocmopiler main purpose is decompile off-the-shelf compilers that use optimization techiniques in the compilation process. They do not aim to handle hand written assembly, nor do they try to outperform existing decompilers. Mjor limitation they face is thet as the length of an input
increases, there is a higher chance that the decompiled code would fail. As the fields of NMT evolves to better handle long inputs, so would the resulted output. Finally, the decompilation testing was implemented on LLVM IR
and x86 assembly to C.

\subsubsection{N-Bref}

This is the current state of the art neural to neural system for decompiling binaries with an \verb|x86| target back to high level \verb|C|.
Essentially, the paper proposes 2 models that support each other.
One model creates an abstract syntax tree (AST) from the disassembly and another that extracts a graph neural network from the assembly, to create a graph of how the low level registers interact with each other.
This way we have a way to inspect how the small scale variables interactions as well as the larger scope and complexity of the application as a whole.

The low level code, in this case the \verb|x86| assembly, is analyzed instruction by instruction to decompose it into its elemental components, the opcode and the registers it is manipulating identifying whether they act as sources or destinations.
We can then create a graph of these relations how the instructions flow,
From this we also try to extrapolate small scale data flows, such as adding directional edges to convey the equivalence of certain nodes and registers to make it easier for the model to recognize the flow of data.
All this is to have an idea of what the paper likes to call "Control Flow".

The heart of the system is the structural transformer.
First it takes the abstract syntax tree and encodes it along with the graph of the assembly instructions to predict what node could be missing from the AST.
Then this new AST is used for another iteration of predicting the next missing node.
This leads to the AST being developed breadth first.
This process concludes when the predicted node leads to termination.

To address the lack of determinism in the structures expressed in abstract syntax tress, all unary operations are converted to conventional mathematical expressions and all while loops are made into for loops.
This however does hurt the final accuracy since the original code is likely to have been written with a variety of styles from different users.
Various hyper parameters are used to fine tune the model, namely they influence how complex the code to be analyzed is expected to be and compensates accordingly.
The dataset this papers used were mostly unoptimized. Their approach was not mature enough to extract the data types of optimized and stripped data structures.

This system can be thought of as an encoder decoder transformer network where both the assembly and the abstract syntax tree are encoded to decode into a more elaborate AST.

The actual architecture of the system is a bit similar to natural language processors.
Both extracted structures are encoded fed into self attention layers and decoded into the new AST node.

However the inner workings of the model also involves more advanced techniques such as memory augmentation and graph augmentation.
Graph augmentation is implemented by making the matrices graphs between attention layers to allow more complicated inferences to be performed.

\subsubsection{Neutron: an attention-based neural decompiler}

Neutron is a neural decompilation framework that uses previous Neural Machine Translation (NMT) models to overcome the bottleneck of decompilation technologies that rely on experts to write rules resulting in low scalability, development difficulties and long cycles. Neutron’s framework is phased into three stages: Code Preprocessing, Neural Translation and Function Reconstruction.

The main goal in the first phase (code pre-processing) is to standardize the PL to assist the model in learning the conversion rules between the tokens of the high-level language to that of the lower-level representation of the language. In order to minimize the complexity of the model, the binary code is disassembled; the assembly language is then used as the target low-level PL. This helps reduce the model difficulty learning conversion rules because it contains richer semantic and structural information. The standardization module focuses on processing the identifiers, numbers, etc. in the PL code, which facilitates model training and translation to better learn the conversion rules between low-level and high-level PLs.

The second stage develops a neural decompilation model that performs the actual recovery of an equivalent high-level C from the low-level target. Afterwards, Neutron trains the model by the name of AsmTran to translate low-level PL into a high-level PL while keeping their functionally similar. This model is based on LSTM-Seq2Seq-attention (Luong et al. 2015) architecture. The AsmTran model is mainly divided into two sub-models. The first sub-model is a text classification model aiming to fine-grain code segmentation for low-level PL based on basic operations. The second sub-model is an NMT model which takes each basic operation of the target low-level PL as input and outputs its corresponding high-level PL. It is worth mentioning that the authors of [] introduce the iterative error correction (EC) mechanism in both sub-models in the training phrase. The prediction errors in the two sub-models’ output are fed back to the sub-model itself to improve the AsmTran’s performance through the manual definition of judgment and EC rules.

In Neutron’s final phase, Function Reconstruction works with the aim of reconstructing the missing dependencies from the previous stage. This includes data flow recovery, control flow recovery, as well as parameters and return value recovery, and it works by the manual definition of rules as it aids in the reconstruction process. Neutron is implemented on the base of the attention-based NMt mechanism in the tensor2tensor framework. The results show that Neutron achieves an average accuracy of 96.96\% on three real-word projects and three different tasks.

Neutron’s contribution is mainly its introduction to a new technique that has high applicability and redability as measured by benchmarks in varying levels of complexity, and it offers a new understanding of the feasibility of applying NMT model that works on natural language to Pls. Similar to natural language translation, the decompilation of low-level PL to high-level PL can also be seen as a translation problem between two natural languages. The results on three real-world projects and three different tasks show that Neutron’s accuracy can reach 96.96\% on average. Compared with using the LSTM-Seq2Seq-attention model (Luong et al. 2015), our approach achieves 36.11\% higher accuracy on average, which reflects that the attention- based NMT method fails to learn the conversion rules between PL pairs effectively. Besides, the proposed approach acheived 74.71\% improvement, on average, compared to using other LSTM models. Even though, the results shown by Neutron are promising, it still exhibits a number of limitations. Firstly, it does not effectively restore the semantics of target low-level PL. It also performs poorly in optimized code primarily due to the adoption of the slicing mechanism. The final limitation is Neutron’s inability to recover user-defined data types such as classes and structures which hinders the readability of the decompiled high-level PL.


\subsection{Evolutionary Decompiler}

\subsubsection{Evolving an Exact Decompiler}

This paper takes a rather unique approach where it considers the compiler itself to be black box that produces binary files to arbitrary text inputs and searches for a corresponding pattern.
This approach guarantees that the resulting code generated will be syntactically and functionally correct.
They call this technique Binary Equivalent Decompilation (BED).

BED starts with random code samples from its extensive database of code then and compiles it.
with Byte-similarity is the comparison metric, it compares this compiled target with the binary it is trying to decompile.
According to that, it mutates the code snippets and consults the database to create a version that has the potential to be more similar to the original binary.
We keep iterating through this process until the result of BED is byte for byte identical to target binary.
This can be considered as an evolutionary algorithm.

Although the results of the paper greatly improved on readability and correctness measures compared to other existing decompilers, a number of limitations hindered the applicability of this framework to real-world applications.
The three biggest problems with this technique is how certain compilers can inheritley create different binary from the same code and optimization level, meaning even if BED happened to achieve identical code, it would still reject the binary.
Another major problem is its lack of scaling for complex and large code bases; it would take an exponential amount of time for it to decompile as the binary grows.
Finally the last fundamental problem is if the code being predicted is not a snippet from the dataset then it would be very difficult to properly decompile it and it is too unlikely for mutation to happen upon the right changes.
This culminates in this solution working well on very small examples but having problems generalizing especially at scale.



\section{Discussion}



\printbibliography

\end{document}
